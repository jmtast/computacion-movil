\documentclass{beamer}

\usetheme[secheader]{Boadilla}
\usecolortheme{seahorse}
\usepackage[spanish]{babel} 		% {Con estos dos anda
\usepackage[utf8]{inputenc} 		% todo lo que es tildes y ñ}
\usepackage{graphicx}

\title{Towards Verifying Android Apps for the Absence of No-Sleep Energy Bugs}
\author{Martín Carreiro - Pablo Rago - Juan Manuel Tastzian}
\date{21 de Mayo de 2014}
\institute[2014]{Facultad de Ciencias Exactas y Naturales}

\begin{document}

\frame{\titlepage}

\section[Temas a desarrollar]{}
\frame{\tableofcontents}

\section{Introducción}

\frame {
	\frametitle{Ventajas de los smartphones actuales}
	Muchas características al alcance de la mano:
	\begin{itemize}
		\item<2-> Procesadores rápidos
		\item<3-> Pantallas grandes
		\item<4-> Cámara, GPS, 3G, Wifi...
	\end{itemize}
}

\frame {
	\frametitle{¿Y antes?}
	\begin{itemize}
		\item[] <1-> Pero...
		\item[] <2-> ...¿se acuerdan del pasado?
	\end{itemize}
}

\frame {
	\frametitle{¡LEYENDA!}
	\begin{center}
		\includegraphics[scale=0.5]{./imgs/nokia1100.jpg}	
	\end{center}
}

\frame {
	\frametitle{¿Cuánto les duraba la batería del Nokia 1100 y similares?}
	\begin{itemize}
		\item<2-> ¿15 hs?
		\item<3-> ¿2 días?
		\item<4-> Como 4 o 5, tranqui!
	\end{itemize}
}

\frame{
	\frametitle{Eso ya no pasa con los Smartphones}
	\begin{itemize}
		\item[] <2-> Lamentablemente, todas las ventajas mencionadas antes necesitan mucha energía para funcionar.
		\item[] <3-> Y por eso, la batería de nuestros celulares de hoy dura entre 12 y 16 horas, promedio, siendo unas 18 o 20 horas una excelente duración de batería (ni hablar de más tiempo).
	\end{itemize}
}

\section{Manejo de energía en Android}

%\subsection{Wakelock API}

\frame {
	\frametitle{Manejo de energía en Android}
	\begin{itemize}
		\item[] <1-> ¿Cómo maneja la energía Android?
		\begin{itemize}
			\item<2-> De manera \textbf{agresiva}.
		\end{itemize}
		\item[] <3-> ¿Le pega para que se porte bien?
		\begin{itemize}
			\item<4-> No \textbf{tan} agresiva, pero le corta el chorro a todo en el momento \textbf{inmediato} en el que se deja de usar.
		\end{itemize}
	\end{itemize}
}

\frame {
	\frametitle{Manejo de energía en Android}
	\begin{itemize}
		\item[] <1-> Pero, ¿y si, como desarrollador, quiero tener el procesador corriendo para recibir algún update o notificación?
		\begin{itemize}
			\item<2-> Ahí es donde entra en juego la \textbf{Wakelock API}.		
		\end{itemize}
	\end{itemize}
}

\frame {
	\frametitle {Manejo de energía en Android}
	\begin{itemize}
		\item[] <1-> Lamentablemente, la complejidad del sistema operativo y los errores que pueden cometer los desarrolladores hacen que el uso inapropiado de Wakelocks se manifiesten como \textit{no-sleep-bugs}.
		\item[] <2-> Los autores del paper decidieron intentar mitigar el problema implementando una herramienta que verifica la ausencia de estos bugs con respecto a una serie de políticas específicas sobre los Wakelocks, utilizando un framework de flujo de datos para analizar las aplicaciones.
		\item[] <3-> Pero...
		\item[] <4-> ¿qué es la Wakelock API?
	\end{itemize}
}

\frame {
	\frametitle {Wakelock API}
	\begin{itemize}
		\item <2-> Permite a los desarrolladores dar directivas específicas sobre los recursos al sistema operativo, ya que Android pone todo en sleep mode ni bien se ponen en estado idle (reposo).
		\item <3-> Es una forma de decir ``esto no me lo apagues" (GPS, pantalla, CPU, etc.).
		\item <4-> Entonces si necesito que algo en particular esté encendido en un momento crítico, pido un Wakelock sobre el mismo para que no se apague, lo uso, y cuando termino, lo libero.
		\item <5-> Genial! Pero ¿qué pasa si me olvido de liberarlo?
		\item <6-> Buena pregunta... ¿cuánto se acuerdan de Orga 2?
	\end{itemize}
}

\frame {
	\frametitle {Repaso de Orga 2}
	\begin{itemize}
		\item <2-> En Orga 2 me enseñaron que debo hacer un free por cada malloc...
			\begin{itemize}
				\item <3-> ...sino \tiny me pegaban.
			\end{itemize}
		\item <4-> Hablando en serio: si no liberan el Wakelock, pasa lo que esperan.
		\item <5-> El recurso no se libera, queda activo, y se \textbf{gasta batería} innecesariamente.
		\item <6-> Esto es a lo que en el paper se lo llama \textit{no-sleep bug}.
		\item <7-> Vendría a ser la versión de ``power management" de los ``memory leaks".
	\end{itemize}
}

%{
%	This text will stay on all pages.
%	\only<1>{
%		\begin{itemize}
%			\item<1->This will only appear on the first page
%			\item<1->This is also only for the first page
%		\end{itemize}
%	}
%	\only<2>{
%		\begin{itemize}
%			\item<2->This will only appear on the second page
%		\item<2->This is also only for the second page
%		\end{itemize}
%      }
%}
%
\subsection{Planteo del paper}

\frame {
	\frametitle{Planteo del paper}
	\begin{itemize}
		\item[] <2-> Los autores del paper desarrollaron una herramienta que permite verificar la ausencia de estos bugs con respecto a una serie de \textbf{políticas específicas sobre los Wakelocks}, utilizando un framework de flujo de datos para analizar las aplicaciones.
		\item[] <3-> Lo peor de todo esto, es que estos bugs son \text{muy difíciles de detectar}, ya que no hace que la app funcione mal o crashee, sino que te reduce la duración de batería del equipo, pasando casi desapercibida como causante de dicho problema.
		\item[] <4-> Pero para entender todo esto, primero hay que entender un poco como funciona el sistema operativo Android por detrás.
	\end{itemize}
}

\frame{
	\frametitle{Un poco de contexto}	
	\begin{itemize}
		\item[] <1-> \textbf{Manejo de energía}
			\begin{itemize}
				\item <2-> Creación, adquisición y liberación de objetos \textbf{Wakelock}.
				\item <3-> Asociados a un recurso particular (CPU, pantalla, GPS, etc.).
				\item <4-> En manos del desarrolador.
			\end{itemize}
	\end{itemize}
	\begin{itemize}
		\item[] <5-> \textbf{Componentes de aplicación}
			\begin{itemize}
				\item <6-> Una aplicación de Android se construye con diversos componentes. Hay 4 tipos principales.
				\begin{itemize}
					\item <7-> Activities
					\item <8-> Services
					\item <9-> BroadcastReceivers
					\item <10-> ContentProviders
				\end{itemize}
				\item <11-> Están ligados entre sí mediante un \textit{lifecycle} o ciclo de vida (visto en clase).
				\item <12-> El desarrollador especifica las acciones a tomar en cada paso del ciclo, implementando un conjunto de \textbf{callbacks}.
				\item <13-> Veamos que hace cada componente.
			\end{itemize}
	\end{itemize}
}

\frame{
	\frametitle{Activities}
	\begin{itemize}
		\item <1-> Proveen la UI de la app.
		\item <2-> Forman un stack, formado de la siguiente manera:
			\begin{itemize}
				\item <3-> \textbf{Running:} La aplicación en \textit{foreground} (visible, activa).
				\item <4-> \textbf{Paused:} La aplicación no está en \textit{foreground}, pero está visible (inactiva).
				\item <5-> \textbf{Stopped:} La aplicación está en \textit{background} (no visible e inactiva).
			\end{itemize}
	\end{itemize}
	\begin{itemize}
		\item[]<6-> \textbf{Callbacks}
			\begin{itemize}
				\item<7-> \textbf{onPause} y \textbf{onResume}: entra y sale del estado \textit{running}.
				\item<8-> \textbf{onStart} y \textbf{onStop}: entra y sale del estado \textit{paused}.
				\item<9-> \textbf{onCreate} y \textbf{onDestroy}: arranca y termina el \textit{lifecycle}.
				\item<10-> \textbf{onRestart}: reinicia una actividad previamente detenida.
			\end{itemize}
	\end{itemize}
}

\frame{
	\frametitle{Services}
	\begin{itemize}
		\item<1-> Realizan operaciones de duración prolongada sin interacción del usuario.
		\item<2-> Se inician con \textbf{startService} y permiten ligarse a ellos con \textbf{bindService}.
		\item<3-> Se configuran durante el \textbf{onCreate}.
		\item<4-> Se ejecuta \textbf{onStartCommand} cuando un servicio es iniciado por otro componente
		\item<5-> Se ejecuta \textbf{onBind} cuando el primer cliente del servicio se liga
		\item<6-> Se ejecuta \textbf{onUnbind} cuando el último cliente del servicio se desliga.
		\item<7-> Se utiliza \textbf{IntentService} para manejar pedidos asincrónicos bajo demanda.
		\item<8-> Al finalizar los callbacks \textbf{onStartCommand}, \textbf{onUnbind} y \textbf{onHandleIntent}, la tarea asociada debería haber terminado, por lo que no debería mantenerse ningún Wakelock.
	\end{itemize}
}

\frame{
	\frametitle{BroadcastReceivers}
	\begin{itemize}
		\item<1-> Responden a anuncios que tienen como alcance a todo el sistema.
			\begin{itemize}
				\item<2-> Producidos por el mismo sistema (por ejemplo, batería baja)
				\item<3-> Desde otras aplicaciones (alertar que ocurrió un evento, por ejemplo)
			\end{itemize}
		\item<4-> Empiezan y terminan su trabajo dentro del llamado del callback \textbf{onReceive}, por lo que tampoco debería mantenerse ningún Wakelock al terminar.
	\end{itemize}		
}

\frame{
	\frametitle{ContentProviders}
	\begin{itemize}
		\item<1-> Dan una forma de encapsular un set de datos estructurado.
		\item<2-> Cada callback es típicamente una unidad de trabajo, por lo que todos los locks deben ser liberados al finalizar el llamado.
	\end{itemize}
}

\frame{
	\frametitle{Bonus: Intent-based Component Comunication}
	\begin{itemize}
		\item<1-> Los componentes se pueden comunicar via mensajes asincrónicos llamdos \textbf{Intents}, que ofrecen una conexión entre componentes de la misma o de distintas aplicaciones.
		\item<2-> Pueden ser explícitos (apuntando a un componente por su nombre) o implícitos (apuntando a una acción a realizar).
		\item<3-> Es útil contabilizar de manera precisa la comunicación entre componentes, ya que es común que se adquieran Wakelocks en un componente y se liberen con otro invocado de forma asincrónica con un intent.
	\end{itemize}		
}














\end{document}